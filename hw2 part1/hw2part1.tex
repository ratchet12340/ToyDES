\documentclass[11pt]{article}
\title{Cryptography \& Network Security I - Fall 2018\\ \large Homework 2 Theory Part A}
\author{Charles Schmitter}
\date{October 9, 2018}

\usepackage[document]{ragged2e}
\usepackage[margin=1in]{geometry}
\usepackage{amsmath}
\newcommand{\tab}{\hspace{10mm}}
\newcommand{\gap}{\vspace{3mm}}
\newcommand{\bigGap}{\vspace{10mm}}

\begin{document}
\maketitle

Q1. Prove that:

\gap{}

(a) $a\equiv b(\mathbf{mod}n)$ implies $b\equiv a(\mathbf{mod} n)$.

\gap{}

We begin with assuming $a\equiv b(\mathbf{mod}n)$. This means $a-b$ is a multiple of $n$. In math terms this is equivalent to: $a-b=c*n$ where $c$ is a constant in the set of all integers. Now, we must prove $b-a$ is a multiple of $n$ to solve the proof. We rearrange $a-b=c*n$ to $b-a=-c*n$, again where $-c$ is  in the set of all integers. Therefore, we have proved $b-a$ is a multiple of $n$, which is equivalent to $b\equiv a(\mathbf{mod} n)$. Thus, our proof has been complete: $a\equiv b(\mathbf{mod}n)$ does in fact imply $b\equiv a(\mathbf{mod} n)$.

\gap{}

(b) $a\equiv b(\mathbf{mod}n)$ and $b\equiv c(\mathbf{mod}n$) implies $a\equiv c(\mathbf{mod} n)$.

\gap{}

We begin by following some of the logic from part (a) and assume that $a-b$ and $b-c$ are multiples of $n$. This can be written in mathematical terms as $a-b=x*n$ and $b-c=y*n$ where $x$ and $y$ are both in the set of all integers. For this proof to be completed, we must prove that $a-c$ is a multiple of $n$. We can substitute in the aforementioned equations into $a-c$ like so: $a-c=(x*n+b) - (b-y*n)$. If we reduce this equation down, we find that $a-c=(x+y) * n$ where $x+y$ is in the set of all integers. Therefore, $a-c$ is a multiple of n, which is equivalent to $a\equiv c(\mathbf{mod} n)$. Thus, we have proven that $a\equiv b(\mathbf{mod}n)$ and $b\equiv c(\mathbf{mod}n$) implies $a\equiv c(\mathbf{mod} n)$.

\bigGap{}

Q2. Using extended Euclidean algorithm to find the multiplicative inverse of:

\gap{}

(a) 1234 mod 4321:\\

\gap{}

\begin{center}
\begin{tabular}{ |c|c| }
\hline
Euclidean Algorithm & Solved For Remainders\\
\hline
$4321 = 1234(3) + 619$ & $619 = 4321 - 1234(3)$\\
$1234 = 619(1) + 615$ & $615 = 1234 - 619(1)$\\
$619 = 615(1) + 4$ & $4 = 619 - 615(1)$\\
$615 = 4(153) + 3$ & $3 = 615 - 4(153)$\\
$4 = 3(1) + 1$ & $1 = 4 - 3(1)$\\
$3 = 1(3)$ & $N/A$\\
\hline
\end{tabular}
\end{center}

\gap{}

Therefore, $gcd(1234, 4321)=1$. Now, on to the extended Euclidean algorithm. We start with the last equation under the `Solved for Remainders' column, substitute in $3=615 - 4(153)$ for 3 in that equation, and combine like terms. We continue this process until we find the following equation: $1=309(4321)-1082(1234)$. We can then deduce that the multiplicative inverse is $-1082$. We can $\mathbf{mod}$ this by $4321$ to retrieve the postiive answer: $3239$. This checks out as $1234 * 3239 \mathbf{mod} 4321 \equiv 1$.

\gap{}

(b) 24140 mod 40902:\\

\gap{}

\begin{center}
\begin{tabular}{ |c| }
\hline
Euclidean Algorithm\\
\hline
$40902 = 24140(1) + 16762$\\
$24140 = 16762(1) + 7378$\\
$16762 = 7378(2) + 2006$\\
$7378 = 2006(3) + 1360$\\
$2006 = 1360(1) + 646$\\
$1360 = 646(2) + 68$\\
$646 = 68(9) + 34$\\
$68= 34(2)$\\
\hline
\end{tabular}
\end{center}

Since 34 is the GDC, and not 1, no multiplicative inverse exists.

\gap{}

(c) 550 mod 1769:\\

\gap{}

\begin{center}
\begin{tabular}{ |c|c| }
\hline
Euclidean Algorithm & Solved For Remainders\\
\hline
$1769 = 550(3) + 119$ & $119 = 1769 - 550(3)$\\
$550 = 119(4) + 74$ & $615 = 1234 - 619(1)$\\
$119 = 74(1) + 45$ & $4 = 619 - 615(1)$\\
$...$ & $...$\\
$16 = 13(1) + 3$ & $3 = 16 - 13(1)$\\
$13 = 3(4) + 1$ & $1 = 13 - 3(4)$\\
$3 = 1(3)$ & $N/A$\\
\hline
\end{tabular}
\end{center}

\gap{}

Therefore, $gcd(550, 1769)=1$. Now, on to the extended Euclidean algorithm. We follow the process from before; start with the last equation under the `Solved for Remainders' column, substitute the second to last equation in, combine like terms, and repeat. We continue this process until we find the following equation: $1=-27(550)+8(1769)$. We can then deduce that the multiplicative inverse is $-27$. We can $\mathbf{mod}$ this by $1769$ to retrieve the postiive answer: $1742$. This checks out as $550 * 1742\mathbf{mod} 1769\equiv 1$.

\bigGap{}

Q3. Determine which of the following are reducible over $GF(2)$:\\

\gap{}

(a) $x^3 + 1$:\\
$f(0)=1\%2=1$\\
$f(1)=2\%2=0$\\

Therefore, (a) is reducible.

\gap{}

(b) $x^3+x^2+1$:\\

$f(0)=0+0+1=1\%2=1$\\
$f(1)=1+1+1=3\%2=1$\\

Therefore, (b) is irreducible.

\gap{}

(c) $x^4+1$:\\

$f(0)=0+1=1\%2=1$\\
$f(1)=1+1=2\%2=0$\\

Therefore, (c) is reducible.

\bigGap{}

Q4. Determine the GCD of the polynomials:\\

\gap{}

(a) $x^3-x+1$ and $x^2+1$ over $GF(2)$

\gap{}

We will denote the first function as $f(x)$ and the second as $g(x)$.
\begin{center}
\begin{tabular}{ |c|c| }
\hline
$x$ & $f(x)$\\
\hline
$0$ & $1$\\
$1$ & $1$\\
\hline
\end{tabular}
\begin{tabular}{ |c|c| }
\hline
$x$ & $g(x)$\\
\hline
$0$ & $1$\\
$1$ & $2\%2=0$\\
\hline
\end{tabular}
\end{center}
We can see that $f(x)$ has no roots, and thus no factors, meaning the GCD between these two polynomials is $1$.

\gap{}

(b) $x^5 + x^4 + x^3 - x^2 - x + 1$ and $x^3+x^2+x+1$ over $GF(3)$

\gap{}

We will denote the first function as $f(x)$ and the second as $g(x)$.
\begin{center}
\begin{tabular}{ |c|c| }
\hline
$x$ & $f(x)$\\
\hline
$0$ & $1$\\
$1$ & $2$\\
$2$ & $51\%3=0$\\
\hline
\end{tabular}
\begin{tabular}{ |c|c| }
\hline
$x$ & $g(x)$\\
\hline
$0$ & $1$\\
$1$ & $4\%3=1$\\
$2$ & $15\%3=0$\\
\hline
\end{tabular}
\end{center}
We can see that $f(x)$ and $g(x)$ share the root $x=2$, so they have a common factor of $x-2$. Converted to the finite field $GF(3)$ gives us the factor $x+1$. The GCD of these two polynomials over $GF(3)$ is $x+1$.

\bigGap{}

Q5. Find $H(K|C)$ of the cryptosystem:\\

\gap{}

First, we find the probabilities of each ciphertext ($Pr(C)$). Then we can compute the probabilities of each ciphertext given a key ($Pr(C|K)$). Then, we can compute the probabilities of each key given a ciphertext ($Pr(K|C)$). Finally, we can compute entropy as the negation of the summation of $Pr(C) * Pr(K|C) * log_2(Pr(K|C))$. This comes out to be:\\
\begin{center}
$\frac{1}{2} * (\frac{3}{4}log_2(\frac{3}{4}) + \frac{1}{4}log_2(\frac{1}{4}))$\\
$+ \frac{1}{4} * (\frac{1}{2}log_2(\frac{1}{2}) + \frac{1}{4}log_2(\frac{1}{4}) + \frac{1}{4}log_2(\frac{1}{4}))$\\
$+ \frac{1}{8} * (\frac{1}{2}log_2(\frac{1}{2}) + \frac{1}{2}log_2(\frac{1}{2}))$\\
$=-0.4055-0.375-0.125=-0.9055$
\end{center}

By applying the final negation, we receive our final answer: $0.9055$.

\end{document}