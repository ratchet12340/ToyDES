\documentclass[11pt]{article}
\title{Cryptography \& Network Security I - Fall 2018\\ \large Make-Up Exam Written Answers}
\author{Charles Schmitter}
\date{October 29, 2018}

\usepackage[document]{ragged2e}
\usepackage[margin=1in]{geometry}
\usepackage{amsmath,amsfonts,amssymb}
\newcommand{\tab}{\hspace{10mm}}
\newcommand{\gap}{\vspace{3mm}}
\newcommand{\bigGap}{\vspace{10mm}}

\begin{document}
\maketitle

Q1.\\

For this question, the collision attack was implemented in \textit{question1.py}. In short, this implementation demonstrates the possibility of an attacker knowing the patterns of a hash function. In particular, the hash function that I implemented builds a hash value based on every fourth character in a string. Once an attacker finds this pattern, he/she can manipulate data to abuse hash collisions.\\

\gap{}

For example, in a message passing example, User A may wish to communicate to User B to meet up somewhere at a specified time. An adversary, User C, can intercept this message through a man-in-the-middle attack, and manipulate this message to have the same hash as the original message, but with different message content. For example, in my sample implementation, the messages `meet5pm!' and `dontcome' will produce a hash collision. Therefore, unsuspecting User B will assume that the imposter message came from User A since the hash value is correct.\\

\gap{}

This pattern of manipulating data around a known pattern of a hash function exists in commonly used hash functions such as MD5 and SHA-1. These patterns can even be faster to perform than a birthday attack. In the case of MD5, there exists the following pattern:\\
If $MD5(A)=MD5(B)$, then $MD5(A+C)=MD5(B+C)$ where $+$ represents concatenation, and A, B, and C are inputs to the MD5 hash function. Therefore, if one can alter the first section of the original message to his/her liking, and then alter it to produce the same hash value as the original, then one can simply add the rest of the original message since it is a common suffix to both the original and the altered message.

\bigGap{}

Q2. \\

\gap{}

10.12\\

\begin{center}
\begin{tabular}{ |c|c|c| } 
 \hline
 X & output & Y\\ 
 \hline
 0 & 6 & none\\ 
 1 & 8 & none \\ 
 2 & 5 & 4, 7 \\ 
 3 & 3 & 5, 6 \\ 
 4 & 8 & none \\ 
 5 & 4 & 2, 9 \\ 
 6 & 8 & none \\ 
 7 & 4 & 2, 9 \\ 
 8 & 9 & 3, 8 \\ 
 9 & 7 & none \\ 
 10 & 4 & 2, 9 \\ 
 \hline
\end{tabular}
\end{center}

Therefore, the points are: $(2,4); (2,7); (3,5); (3,6); (5,2); (5,9); (7,2); (7,9); (8,3); (8,8); (10,2); (10,9);$

\gap{}

10.13\\

To find the negative of a point $P(X_P, Y_P)$, we can use the equation: $-P = (X_P, -Y_P)$. Once we have negated $-Y_P$, we use modulo $17$ since we are in $Z_17$. Thus, we can map each given point to its negative:\\
$P(5,8) \rightarrow -P(5,9)$\\
$Q(3,0) \rightarrow -Q(3,0)$\\
$R(0,6) \rightarrow -R(0,11)$\\

\gap{}

10.14\\

Given $G(2,7)$, we first find $2G=(10, 0)$. When we attempt to compute $3G=2G+G$, we find that $3G$ is equivalent to $(2,7)$. Therefore, we can deduce that for any $kG$ where $k$ is odd, $kG=(2,7)$, and for any $kG$ where $k$ is even, $kG=(10,0)$. So:\\
$1G=(2,7)$\\
$2G=(10,0)$\\
$3G=(2,7)$\\
$4G=(10,0)$\\
$5G=(2,7)$\\
$6G=(10,0)$\\
$7G=(2,7)$\\
$8G=(10,0)$\\
$9G=(2,7)$\\
$10G=(10,0)$\\
$11G=(2,7)$\\
$12G=(10,0)$\\
$13G=(2,7)$\\

\gap{}

10.15\\

We first compute B's public key with $P_B=n_b * G$. Here, we notice we have $7(2,7)$ in the same field as the previous problem. Thus, we can deduce that $7(2,7)$ is equivalent to $(2,7)$. We can then compute the ciphertext with $C_M=\{kG, P_M+kP_B\}$. We then find $C_M=\{(2,7),(8,8)\}$. For (c) of this problem, we must now compute the decryption of $C_M$. We therefore must compute $(8,8)-7(2,7)$, where the $7$ is the private key of B. This subtraction is equivalent to $(8,8)+(2,4)$. We can do a simple ECC point addition, and our result is $(10,9)$, thus revealing the original plaintext as given by the problem.

\bigGap{}

Q3. \\

\gap{}

For this question, the Miller-Rabin and Pollard-Rho algorithms were implemented (found in \textit{question3.py}). The answers found with this script are written below.

\gap{}

(a) Of the four numbers given:\\
	The prime numbers are: $31531; 485827; 15485863$\\
	The composite number is: $520482$\\
	
\gap{}
	
(b) The composite number $520482$ has factors: $2; 3; 223; 389$


\bigGap{}

\end{document}