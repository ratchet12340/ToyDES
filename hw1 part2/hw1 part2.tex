\documentclass[11pt]{article}
\title{Cryptography \& Network Security I - Fall 2018\\ \large Homework 1 Part 2}
\author{Charles Schmitter}
\date{September 24, 2018}

\usepackage[document]{ragged2e}
\usepackage[margin=1in]{geometry}
\usepackage{amsmath}
\newcommand{\tab}{\hspace{10mm}}
\newcommand{\gap}{\vspace{5mm}}
\newcommand{\bigGap}{\vspace{10mm}}

\begin{document}
\maketitle

Q1. For the simplified DES, consider Sbox $S_0$ and show how DiffCrypto attack would work.

\gap{}

A.\\
\tab With the simplified DES, at the step where Sbox $S_0$ is used, there exists $S0_E$ which is the expanded 4-bit input (expanded to 8-bits) and the 8-bit subkey $S0_K$. These two values are XORed with one another to produce the input to Sbox $S_0$ as well as Sbox $S_1$. Since we are required to only look at Sbox $S_0$, we will only look at the first 4 bits of this XORed value (these first 4 bits are the bits that go on to Sbox $S_0$.

\tab For a differential cryptanalysis attack to work, we must first construct the differential distribution table for $S_0$. Once constructed, we can consider a particular input XOR to evaluate. We will take a look at the input XOR value of $F$. The following pairs of values XOR to $F$:

\begin{center}
$(15, 0); (10, 5); (12, 3); (9, 6); (14, 1); (7, 8); (2, 13); (4, 11)$
\end{center}

After putting each pair through Sbox $S_0$, we find the following mappings of pairs to output XORs:
\begin{itemize}
\item $(9, 6) \rightarrow 1$
\item $(4, 11) \rightarrow 2$
\item $(15, 0); (10, 5); (12, 3); (14, 1); (7, 8); (2, 13) \rightarrow 3$
\end{itemize}

With this in mind, suppose we have two inputs to $S_0$: $12$ and $3$ (which XOR to $F$) and the output XOR $1$. We also know the following:

\begin{center}
$S0_K = S0_I \oplus S0_E$
\end{center}

With all of this known, we can then find:

\begin{itemize}
\item $12 \oplus 9 = 5$
\item $12 \oplus 6 = 10$
\item $3 \oplus 9 = 10$
\item $3\oplus 6 = 5$
\end{itemize}

We therefore know potential keys are in the set:
\begin{center}
$\{5, 10\}$
\end{center}

We can do this process again. Suppose we have two inputs to $S_0$: $10$ and $5$ (which again XOR to $F$) and the output XOR $2$. We can then similarly find:

\begin{itemize}
\item $10 \oplus 4 = 14$
\item $10 \oplus 11 = 1$
\item $5 \oplus 4 = 1$
\item $5\oplus 11 = 14$
\end{itemize}

We then end up with the result set:
\begin{center}
$\{1, 14\}$
\end{center}

We take this result set and take the intersection of this set with the result set of the last findings. We then have a set of potential keys:

\begin{center}
$\{1, 5, 10, 14\}$
\end{center}

We can repeat this DiffCrypto process with different input XORs and different output XORs, eventually determining what the key is precisely. This is how a DiffCrypto attack would work, with Sbox $S_0$ being used as an example.

\bigGap{}

Q2. Consider the crypto system and compute $H(K|C)$.

\gap{}

A. \\
\tab To compute $H(K|C)$, we can use the the theorem: $H(K|C) = H(K) + H(P) - H(C)$.

We will start with computing $H(P)$, the entropy of the plaintext component:

\begin{center}
$H(P) = - \sum_{i=1}^{n} p_i * log_2 * p_i $
\end{center}
where $n$ is the number of plaintexts and $p_i$ is the probability of each plaintext. So:
\begin{center}
$H(P) = -[\frac{1}{3} log_2 \frac{1}{3} + \frac{1}{6} log_2 \frac{1}{6} + \frac{1}{2} log_2 \frac{1}{2}] = 1.459$
\end{center}

Next, we will similarly compute the entropy of the key component:

\begin{center}
$H(K) = - \sum_{i=1}^{n} p_i * log_2 * p_i $
\end{center}
where $n$ is the number of keys and $p_i$ is the probability of each key. So:
\begin{center}
$H(K) = -[\frac{1}{2} log_2 \frac{1}{2} + \frac{1}{4} log_2 \frac{1}{4} + \frac{1}{4} log_2 \frac{1}{4}] = 1.5$
\end{center}

Finally, we will compute the entropy of the ciphertext component. To compute the probability of each ciphertext, we will use:
\begin{center}
$P_C(y) = \sum{}{} P_K(k) * P_P(d_k(y))$ where $\{k: y\in C(k)\}$
\end{center}
Below are the probabilties computed for each ciphertext in the set $\{1, 2, 3, 4\}$:
\begin{itemize}
\item $P_C(1) = \frac{7}{24}$
\item $P_C(2) = \frac{5}{12}$
\item $P_C(3) = \frac{3}{24}$
\item $P_C(4) = \frac{1}{6}$
\end{itemize}
Now we can use the entropy formula as used in computing $H(P)$ and $H(K)$:

\begin{center}
$H(C) = -[\frac{7}{24} log_2 \frac{7}{24} + \frac{5}{12} log_2 \frac{5}{12} + \frac{3}{24} log_2 \frac{3}{24} + \frac{1}{6} log_2 \frac{1}{6}] = 1.851$
\end{center}

We will finalize this problem by using the theorem as stated at the start of the solution:
\begin{center}
$H(K|C) = H(K) + H(P) - H(C) = 1.459 + 1.5 - 1.851 = 1.108$
\end{center}
\end{document}